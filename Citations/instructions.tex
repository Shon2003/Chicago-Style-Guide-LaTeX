\documentclass{article}

\usepackage[utf8]{inputenc}
\usepackage[T1]{fontenc}
\usepackage{lmodern}
\usepackage{geometry}
\geometry{margin=1in}
\usepackage{setspace}
\usepackage{listings}
\usepackage{xcolor}

\lstset{%
    language=[LaTeX]TeX,
    basicstyle=\ttfamily\small,
    breaklines=true,
    frame=single,
    numbers=none,
    tabsize=2,
    keywordstyle=\color{blue},
    commentstyle=\color{gray},
    stringstyle=\color{red},
    showstringspaces=false,
}

\title{Citations in Chicago Manual of Style (Notes and Bibliography) with LaTeX and biblatex-chicago}
\author{}
\date{}

\begin{document}

\maketitle

This guide explains how to set up your LaTeX document for Chicago Manual of Style (Notes and Bibliography) citations using the \texttt{biblatex-chicago} package. Below are the detailed instructions, code examples, and compilation steps using XeLaTeX.

\vspace{1em}

\section{Load the Package}

Include the following in your document preamble to load the \texttt{biblatex-chicago} package with the necessary options. The \texttt{notes} option enables Chicago-style footnotes, and this configuration is for the notes and bibliography style. The \texttt{backend=biber} option specifies Biber as the bibliography processor and \texttt{short} enables shortened citations on subsequent uses.

\begin{lstlisting}
\usepackage[notes,backend=biber,short]{biblatex-chicago}
\end{lstlisting}

\vspace{1em}

\section{Add the Bibliography Resource}

Register your bibliography file (e.g. \texttt{references.bib}) in the preamble.

\begin{lstlisting}
\addbibresource{references.bib}
\end{lstlisting}

\vspace{1em}

\section{Citing Sources in Your Document}

Use the \verb|\autocite| command to automatically generate Chicago-style footnotes for your citations (Notes and Bibliography style). The first citation provides a full reference, and subsequent citations use a shortened format if the \texttt{short} option is enabled.

\textbf{Example:}

\begin{lstlisting}
World War I had lasting impacts on Germany\autocite{GermanHistoryVersailles}.
\end{lstlisting}

Another example:

\begin{lstlisting}
Otto Dix depicted war as overwhelming destruction\autocite{UMMAArtilleryBattle}.
\end{lstlisting}

\vspace{1em}

\section{Print the Bibliography}

At the end of your document, include the following command to display the complete bibliography:

\begin{lstlisting}
\printbibliography
\end{lstlisting}

\vspace{1em}

\subsection{Single Spacing in the Bibliography}

To ensure that the bibliography is single-spaced, include the following command before printing the bibliography:

\begin{lstlisting}
\renewcommand{\bibfont}{\setstretch{1}}
\end{lstlisting}

\vspace{1em}

\section{Compilation Instructions with XeLaTeX}

To compile your document using XeLaTeX, follow these steps:

\begin{enumerate}
    \item Run XeLaTeX on your document:
          \begin{lstlisting}[language=bash]
xelatex essay.tex
          \end{lstlisting}

    \item Process the bibliography with Biber:
          \begin{lstlisting}[language=bash]
biber essay
          \end{lstlisting}

    \item Run XeLaTeX again to update references:
          \begin{lstlisting}[language=bash]
xelatex essay.tex
          \end{lstlisting}

    \item Run XeLaTeX one more time to ensure all citations are correctly updated:
          \begin{lstlisting}[language=bash]
xelatex essay.tex
          \end{lstlisting}
\end{enumerate}

\noindent Combined:
\begin{lstlisting}[language=bash]
xelatex essay.tex && biber essay && xelatex essay.tex && xelatex essay.tex
\end{lstlisting}

\subsection{\texttt{compile.zsh} Script to Automate Compiling}
\begin{lstlisting}[language=bash]
echo "------------------------------------------------" | tee compile.log
echo "xelatex essay.tex" | tee -a compile.log
echo "------------------------------------------------" | tee -a compile.log
xelatex -interaction=nonstopmode -file-line-error essay.tex 2>&1 | tee -a compile.log

echo "------------------------------------------------" | tee -a compile.log
echo "biber essay" | tee -a compile.log
echo "------------------------------------------------" | tee -a compile.log
biber essay 2>&1 | tee -a compile.log

echo "------------------------------------------------" | tee -a compile.log
echo "xelatex essay.tex" | tee -a compile.log
echo "------------------------------------------------" | tee -a compile.log
xelatex -interaction=nonstopmode -file-line-error essay.tex 2>&1 | tee -a compile.log

echo "------------------------------------------------" | tee -a compile.log
echo "xelatex essay.tex" | tee -a compile.log
echo "------------------------------------------------" | tee -a compile.log
xelatex -interaction=nonstopmode -file-line-error essay.tex 2>&1 | tee -a compile.log
\end{lstlisting}

\vspace{1em}
\clearpage

\section{Example Bibliography Entries}

Below are sample entries from the \texttt{references.bib} file:

\begin{lstlisting}
@online{UMMAArtilleryBattle,
    author      = {University of Michigan Museum of Art},
    title       = {Artillery Battle (Artillerieschlacht)},
    shortauthor = {UMMA},
    shorttitle  = {Artillery Battle},
    year        = {1917},
    url         = {https://umma.umich.edu/objects/artillery-battle-artillerieschlacht-1967-1-41/},
    urldate     = {2025-02-10},
}

@online{UMMAHunger,
    author      = {University of Michigan Museum of Art},
    title       = {Hunger},
    shortauthor = {UMMA},
    shorttitle  = {Hunger},
    year        = {1924},
    url         = {https://umma.umich.edu/objects/hunger-1957-1-105/},
    urldate     = {2025-02-10},
}

@misc{Hartlaub1925,
    author       = {Gustav Friedrich Hartlaub},
    title        = {Introduction to the \textit{Neue Sachlichkeit} Exhibition Catalogue},
    shortauthor  = {Hartlaub},
    shorttitle   = {Neue Sachlichkeit},
    howpublished = {Exhibition catalog, Kunsthalle Mannheim},
    year         = {1925},
}

@book{Plumb2006,
    author      = {Steve Plumb},
    title       = {{\textit{Neue Sachlichkeit} 1918--33: Unity and Diversity of an Art Movement}},
    shortauthor = {Plumb},
    shorttitle  = {Neue Sachlichkeit 1918--33},
    publisher   = {Rodopi},
    address     = {Amsterdam},
    year        = {2006},
}

@online{GermanHistoryVersailles,
    author      = {German History in Documents and Images},
    title       = {Treaty of Versailles and its Economic Impact},
    shortauthor = {GHDI},
    shorttitle  = {Versailles Economic Impact},
    year        = {1919},
    url         = {https://ghdi.ghi-dc.org/pdf/eng/DEST_SOLMITZ_ENG.pdf},
    urldate     = {2025-02-10},
}
\end{lstlisting}

\vspace{1em}

With these steps, your LaTeX document is configured for Chicago Manual of Style (Notes and Bibliography) citations using the \texttt{biblatex-chicago} package, and you are ready to compile with XeLaTeX.

\end{document}
